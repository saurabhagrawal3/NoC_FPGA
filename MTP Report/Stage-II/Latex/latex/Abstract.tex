% Chapter 1
\chapter*{Abstract}
\hspace{5mm}The emerging new generation of high speed computing / Application Specific Integrated Chip design / System-on-Chip designs consists of devices such as micro-controllers, memories, microprocessors, Digital Signal Processors (DSPs), signal conditioning units, etc. The technology trends toward the increased number of processing elements with higher levels of integration and higher performance. This increase in complexity and packaging of processing elements needs an efficient communication infrastructure within and outside the chip. The Network-on-Chip (NoC) architecture paradigm, based on a modular packet-switched mechanism, can address many of the on-chip communication design issues such as wiring complexity and integration of a large number of Intellectual Property (IP) cores. Apart from the communication infrastructure hardware testing and implementation often fall short of on chip resources. For such architectures we have to utilize multi FPGA implementation of these processing element. When a multi-FPGA implementation is considered the issues that significantly lowers the performance is the interconnect delays and the need for large number of interfacing ports complicates the designers tasks. These issues and drawbacks can be handled by implementing high speed serial or partially serial communication links which reduces the number of interconnects along with providing high speed communication. The challenge for designers is to partition the test application. The solution for which is implementation of Network-on-Chip for communication interface independent of application which has been now widely adopted by the ASIC designers community to overcome application partitioning and communication bottlenecks.